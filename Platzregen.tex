\documentclass[prd,12pt,tightenlines,notitlepage,nofootinbib]{revtex4-1}
%\usepackage[utf8]{inputenc}
\usepackage[ngerman]{babel}
%\usepackage{mathpazo}
\usepackage{fontspec}
\setmainfont[Ligatures=TeX]{TeX Gyre Pagella}
\setsansfont[Ligatures=TeX]{Arimo}
\setmonofont[Ligatures=TeX]{Cousine}
%\fontsize{12}{12}
%\linespread{2.4}
\linespread{1.25}

\begin{document}

\title{Der Platzregen (1953)}
\author{Koreanischer Originaltext von Sun-won Hwang (1915-2000)
\\ Übersetzt ins Deutsche von Jae-hyeon Park
\\ Korrigiert von Jupp Hartmann}
\maketitle

\noindent
Als der Junge am Bach das Mädchen sah, erkannte er sie sofort als die
Urenkelin von Herrn Yun\footnote{Im Originaltext heißt seine Anrede "`Choshi"'.
"`Choshi({\fontspec{Source Han Sans K}初試})"' bedeutet wörtlich "`Erste (Vorbereitungs)Prüfung"' und
bezeichnete auch eine Person, die sie bestand.
Während der Joseon Dynastie gab es
die erste (lokale) und die zweite (zentrale) Vorbereitungsprüfung,
danach drei Stufen Hauptprüfungen für die Auswahl der Beamten.
Eine regelmäßige "`Choshi"' fand alle drei Jahre statt. Dabei wurden
im ganzen Staat 1400 Personen ausgewählt.}.
Sie bewegte ihre Hände im Wasser und hatte
Spaß daran.  Sie hatte wohl noch nie solch einen Bach in Seoul gesehen.
Schon einige Tage hatte sie nach der Schule mit dem Wasser gespielt.
Bis gestern hatte sie sich am Bach amüsiert,
aber heute war sie mitten auf den Trittsteinen.  Er
setzte sich ans Ufer, um zu warten, bis sie wegginge.  Glücklicherweise kam
jemand und sie machte den Weg frei.

Am nächsten Tag ging der Junge ein bisschen später an den Bach.
Mitten auf den Trittsteinen sitzend wusch das Mädchen dieses Mal ihr
Gesicht.  Reinweiß war ihr Nacken gegen die rosa
Strickjacke mit den hochgekrempelten Ärmeln.  Nachdem sie eine Zeit lang
ihr Gesicht gewaschen hatte, sah sie ins Wasser hinein.  Sie starrte wohl das
Abbild ihres Gesichts an.  Plötzlich schöpfte sie mit den Händen
Wasser, vielleicht schwamm ein kleiner Fisch vorbei.  Sie schöpfte nur schnell
Wasser, ohne darauf zu achten, dass er am Ufer saß.  Jedoch sie fing
die ganze Zeit nichts.  Anscheinend machte sie sich einen Spaß daraus,
nur weiter das Wasser zu schöpfen.  Vielleicht würde sie nur weggehen,
wenn jemand den Bach überquerte.  Dann nahm sie etwas aus dem Wasser.
Einen weißen Kieselstein.  Plötzlich stand sie auf und hüpfte über die
Trittsteine.  Auf der anderen Seite des Bachs drehte sie sich um und
sagte: "`Dieser Dumme!"'  Der Kieselstein flog hierhin.  Unbewusst stand
der Junge abrupt auf.  Das Mädchen lief mit flatterndem kurzem
Haar.  Sie trat auf den Weg durch ein Schilffeld.  Dahinter lagen nur
Schilfblüten glänzend in der Herbstsonne\footnote{In Korea ist
der Himmel im Herbst meistens klar.}.
Das Mädchen werde bald aus dem
Schilffeld erscheinen, dachte er.  Eine lange Zeit fühlte er vergehen.
Trotzdem erschien sie nicht.  Er stellte sich auf die Zehen.  Noch einmal
ziemlich viel Zeit sei vergangen, meinte er.  Dort am Ende des
Schilffelds bewegte sich ein Strauß.  Das Mädchen hielt Schilfblüten
im Arm.  Sie ging nun langsam.  Ihr Haar mit einer Schilfblüte dazu
leuchtete vor der extrem klaren Herbstsonne.  Es schien, als würde
nicht das Mädchen gehen, sondern die Schilfblüte.  Der Junge blieb
so stehen, bis die Schilfblüte gar nicht mehr zu sehen war.  Er sah
auf den Kieselstein hinunter, den das Mädchen geworfen hatte.  Die Nässe darauf
hatte sich verflüchtigt.  Er nahm den Kieselstein und steckte ihn in die
Tasche.

Vom nächsten Tag an ging der Junge noch später an den Bach.  Es gab
keine Spur % Schatten
des Mädchens, zum Glück.  Jedoch war es komisch, dass
irgendeine Ecke seines Herzens sich immer leerer anfühlte, als er Tag für
Tag das Mädchen nicht traf.  Es wurde seine Gewohnheit, den
Kieselstein in der Tasche zu berühren.  Eines solchen Tages setzte er
sich mitten auf die Trittsteine, auf denen sitzend das Mädchen sich früher mit
dem Wasser vergnügt hatte.  Er steckte seine Hände ins Wasser.
Er wusch sein Gesicht.  Er sah ins Wasser hinein.  Das dunkelbraune
Gesicht spiegelte sich.  Es gefiel ihm nicht.  Er schöpfte mit den
Händen das Gesicht im Wasser, immer wieder.  Irgendwann wurde er
überrascht und stand auf.  Das Mädchen kam über den Bach her!
"`Sie hat sich versteckt und gesehen, was ich getan habe."'
Er fing an,
zu laufen.  Er tat einen falschen Schritt auf den Trittstein.  Ein
Fuß rutschte ins Wasser.  Er lief weiter.  Er wünschte, er könnte sich irgendwo
verstecken.  Auf dieser Seite gab es kein Schilffeld, sondern ein Buchweizenfeld.
Er dachte, die Buchweizenblüten röchen in der Nase schärfer als je
zuvor.  Die Mitte seiner Stirn schmerzte.  Salzhaltige Flüssigkeit floss ihm
in den Mund.  Nasenbluten.  Er rannte einfach und wischte sich dabei
mit einer Hand das Blut ab.  Es kam ihm vor, als wäre irgendwo eine
Stimme hinter ihm hergelaufen: "`dumm, dumm"'.

Es war Samstag.  Am Bach fand der Junge das Mädchen, das einige
Tage nicht erschienen war.  Sie amüsierte sich auf der anderen Seite sitzend mit dem Wasser.  Er fing an, über die Trittsteine so zu gehen, als hätte er sie
nicht gesehen.  Heute ging er vorsichtig über die Trittsteine, auf denen
er bisher wie auf einem Boulevard gelaufen war, denn er hatte neulich vor dem
Mädchen einen Fehler gemacht.  "`Hallo."'  Er tat so, als hätte er das nicht
gehört.  Er trat aufs Ufer.  "`Hallo, wie heißt diese Muschel?"'
Unwillkürlich drehte er sich um.  Sein Blick traf die klaren schwarzen
Augen des Mädchens.  Er senkte sofort den Blick auf ihre Handfläche.
"`Seidenmuschel\footnote{Direkte Übersetzung vom Koreanischen Namen.
  Der wissenschaftliche Name ist "`Peronidia Venulosa"'.}."'\\
"`Was für ein schöner Name!"'  Sie erreichten eine Weggabelung.
Von hier musste das Mädchen etwa 6 Kilometer nach unten gehen, er musste
etwa 4 Kilometer nach oben.  Sie hielt an und sagte: "`Warst du über den
Berg da?"'  Sie zeigte ans Ende des Felds.\\
"`Nein."'\\
"`Gehen wir mal dorthin?
Ich langweile mich allein zu Tode, nachdem ich aufs Land
gezogen bin."'\\
"`Es ist weiter, als es scheint."'\\
"`Aber nicht zu weit.
Ich habe in Seoul ganz weite Ausflüge gemacht."'  Ihre Augen kamen ihm
vor, als würden sie sofort sagen: "`dumm, dumm!"'  Sie traten auf den Weg
durch ein Reisfeld.  Sie gingen am Ort der Reisernte vorbei.
Da stand ein Strohmann.  Der Junge schüttelte das Strohseil.  Einige
Spatzen flogen weg.  "`Ach, heute müsste ich früh nach Hause gehen, um
Spatzen im Nutzgarten zu beobachten"', erinnerte er sich.\\
"`Wie lustig!"'
Das Mädchen griff und schüttelte unbändig das Seil.  Der Strohmann schwankte und
tanzte weiter.  Sie bekam ein Grübchen in ihrer linken Wange.  Drüben
stand noch ein Strohmann.  Das Mädchen lief dorthin.  Der Junge lief hinterher,
als hätte er versucht, zu vergessen, dass er an einem Tag wie heute
früh nach Hause gehen müsste, um im Haushalt zu helfen.  Er rannte
einfach am Mädchen vorbei.  Grashüpfer stießen gegen sein
Gesicht und juckten.  Der blaue klarste Herbsthimmel kreiste in seinem Blick.
"`Schwindelig, denn der Adler, der Adler, der Adler kreist."'  Der Junge blickte
zurück.  Sie schüttelte den Strohmann, an dem er soeben vorbeikam.  Der
schwankte mehr als der Vorige.  Es gab einen Graben am Ende des
Reisfelds.  Zunächst hüpfte das Mädchen darüber.  Von dort bis zum Fuße
des Berges war ein Feld.  Sie gingen an der Feldgrenze vorbei, wo
Sorghumbündel standen.  "`Was ist das da?"'\\
"`Eine Beobachtungshütte."'\\
"`Schmecken Melonen\footnote{
  Das Obst im Originaltext ist die Variante des wissenschaftlichen Namens
  "`Cucumis melo var.\ makuwa"'.}
hier gut?"'\\
"`Klar!  Melonen schmecken gut aber Wassermelonen schmecken besser."'\\
"`Ich will eine probieren."'  Er
trat ins Rettichfeld ein, wo Melonen erzeugt worden waren.\footnote{
Normalerweise erzeugt man auf dem gleichen Feld
im Sommer Melonen und im Winter Rettiche.}
Er zog
zwei Rettiche heraus und brachte sie her.  Sie waren noch nicht reif.
Nachdem er die Blätter verdreht und weggeworfen hatte, gab er dem Mädchen einen
Rettich.  Er biss ein Stück vom Kopf des Rettichs ab, schälte den
Rettich mit dem Nagel, biss hinein und zeigte ihr wohl, wie man ihn
isst.  Das Mädchen tat das Gleiche.  Jedoch hatte sie noch keine drei
Bissen gegessen, da sagte sie: "`Ach, das schmeckt scharf und stinkt."'  Sie warf den
Rettich weg.
\\ "`Das schmeckt wirklich zu schlecht."'  Er warf seinen auch
weg.  Sie kamen in die Nähe des Berges.  Die Augen brannten ihnen vor dem Herbstlaub.
"`Toll!"'  Das Mädchen lief zum Berg.  Dieses Mal lief er nicht hinterher.
Trotzdem pflückte er in kurzer Zeit mehr Blumen als das Mädchen.
"`Das ist eine Kamille, das ist Buschklee, das ist eine Ballonblume, \ldots"'
\\ "`Ich habe nicht gewusst, dass eine Ballonblume so schön ist.\footnote{
Im Essen findet man oft die Wurzel der Ballonblume.}
Mir gefällt
Violett!  \ldots\  Übrigens, wie heißt diese gelbe Blume, die wie ein
Sonnenschirm aussieht?"'\\
"`Patrinia."'  Sie hielt die Patrinia wie einen
Sonnenschirm, mit einem Grübchen in ihrem etwas roten Gesicht.  Er
bracht noch einen Strauß Blüten.  Er suchte frische Zweige aus und
gab sie ihr.  Doch sie sagte: "`Wirf nichts weg!"'  Sie stiegen auf den
Gebirgskamm.  Auf der anderen Seite des Tals lagen einige Hütten
friedlich zusammen.  Sie setzten sich auf einen Fels, obwohl keiner der beiden
so gesagt hatte.
Sie fühlten, als wäre ihre Nähe besonders ruhig geworden.
Nur die
brennende Herbstsonne verbreitete den Geruch des trocknenden Grases.
"`Welche Blüten sind das da?"'  Pfeilwurzstämme waren auf einem
Steilhang mit den Blüten verwickelt.  "`Sie sehen wie Glyzinien aus.
Es gab große Glyzinien in meiner Schule in Seoul.  Die Blüten da
erinnern mich an meine Freunde, die mit mir unter Glyzinien gespielt
haben."'  Sie stand still auf und ging an den Hang.  Sie versuchte,
einen Stamm mit vielen Blüten abzuschneiden.  Er war nicht leicht
zu schneiden.  Sie nahm alle ihre Kräfte zusammen aber rutschte aus.
Sie fasste den Stamm.  Er war überrascht und lief hin.  Sie streckte
eine Hand aus.  Er hielt ihre Hand und hob sie auf.  Es tat ihm leid,
dass er ihr den Stamm nicht abgeschnitten hatte.  Ein Bluttropfen war auf
dem rechten Knie des Mädchens.  Er legte seine Lippen auf die Wunde und fing an,
zu saugen.  Anscheinend hatte er plötzlich irgendeine Idee.  Er stand abrupt auf
und lief nach drüben.  Nach einer Weile kam er keuchend zurück.  "`Das
wird dich heilen."'  Er strich Kiefernharz auf die Wunde.  Sofort
lief er runter zu den Pfeilwurzstämmen und brachte einige mit vielen
Blüten, die er mit den Zähnen abgeschnitten hatte.  Danach sagte er: "`Da ist ein Kalb.
Gehen wir mal dorthin!"'  Es war ein Gelbes Kalb.  Es hatte noch keinen
Nasenring.  Er hielt es im Zaum und tat, als hätte er es am Rücken
gekratzt.  Plötzlich sprang er darauf.  Es hüpfte herum.
Das weiße Gesicht, die rosa Strickjacke, der Blaue Rock des
Mädchens mischten sich mit den Blüten in ihrem Arm.  Alle sahen wie
ein großer Strauß Blüten aus.  "`Mir ist schwindelig, aber ich steige nicht
ab."'  Er war stolz, dass er konnte, was das Mädchen nicht nachahmen kann.
"`Was macht ihr hier?"'  Ein Bauer kam durch das Pampasgras herauf.  Der
Junge sprang von dem Kalb herunter.  Er hatte Angst, dass der Bauer ihn schelten
würde, weil es dessen Hüftknochen schaden kann, auf ein Kalb zu steigen.  Der
bärtige Bauer blickte jedoch über das Mädchen und band nur den Zaum
auf.  "`Eilt nach Hause!  Vielleicht kommt ein Platzregen."'  Wirklich
war eine platte dunkle Wolke schon da über ihren Köpfen. Aus alle
Richtungen toste es plötzlich.  Ein Wind ging durch und
schüttelte dabei die Blätter.  In einem Augenblick wurde die Umgebung violett.
Während sie den Berg hinuntergingen, erklangen Regentropfen
auf den Eichenblättern.
Die Tropfen waren groß.  Es lief kalt über den Nacken. % hinunter.
Sofort versperrten Regenstreifen die Sicht.  Sie fanden im
Regennebel eine Beobachtungshütte.  Darin suchten sie Schutz vor dem Regen.
Jedoch hatte die Hütte geneigte Säulen und ein zerrissenes
Dach.  Trotzdem fand er einen Raum, wo weniger Regen fiel und ließ das Mädchen
darin eintreten.  Die Lippen des Mädchens wurden ganz Blau.
Sie schüttelte immer wieder die Schultern.  Er zog seine Baumwolljacke
aus und wickelte sie ihr um die Schultern.  Als er so tat, hob sie nur
leise den vom Regen feuchten Blick zu ihm.  Danach sonderte sie von dem
Strauß in ihrem Arm gebrochene Zweige mit zerzausten Blüten aus.
Der Regen fing an, auch dort zu fallen, wo das Mädchen stand.  Da war kein Schutz
mehr.  Er sah hinaus und lief anscheinend mit irgendeiner Idee zum
Sorghumfeld.  Er spaltete ein gestelltes Sorghumbündel,
schaute das Innere hinein und stellte mehr
Bündel darauf, die er aus der Nähe herbeibrachte.  Er sah wieder nach drinnen.
Danach winkte er sie her.  Kein Regen fiel in die Bündel, aber es war
leider dunkel und eng.  Der Junge musste draußen vorn im Regen
sitzen.  Dabei stieg Dampf von seinen Schultern.  Das Mädchen
flüsterte ihm zu, er solle drinnen sitzen.  Er sagte, es gehe so.  Wieder
sagte sie ihm, er solle herein kommen.  Er konnte es nicht mehr vermeiden,
einzutreten und ging nach hinten.
Dabei brach der Strauß im Arm des
Mädchens.  Trotzdem dachte sie, das sei kein Problem.  In ihre Nase drang
der Geruch des vom Regen nassen Körpers des Jungen.  Jedoch wandte
sie den Kopf nicht.  Sie fühlte doch, dass die Körperwärme des Jungen
ziemlich den zitternden Leib heizte.  Die Sorghumblätter hörten
plötzlich auf, Lärm zu machen.  Draußen wurde hell.  Sie kamen aus
den Sorghumbündel.  Vor ihnen glänzte die Sonne.  Sie
erreichten den Graben.  Da strömte viel mehr Wasser.  Das Wasser war
schlammig in ganz roter Farbe.  Sie konnten nicht darüber hüpfen.  Er
drehte dem Mädchen den Rücken.  Darauf stieg sie ohne Bedenken.  Das
Wasser reichte ihm bis zu den hochgekrempelten Kniehosen.  Sie hielt
sich an seinem Nacken fest und stieß dabei einen Schrei aus.  Schon
bevor sie an die andere Seite gelangten, wurde der Herbsthimmel ohne
eine einzige Wolke blau, als wäre er immer so klar gewesen.

Danach zeigte das Mädchen sich nicht mehr.
Er fand sie nicht, obwohl er jeden Tag an den Bach herlief.
An der Schule beobachtete er in der Pause den Spielplatz.
Er blickte verstohlen in die sechste Mädchenklasse.
Trotzdem sah er sie nicht.
Auch an diesem Tag kam er an den Bach und berührte
den weißen Kieselstein dabei.
Dann fand er das Mädchen an diesem Ufer sitzen.
Sein Herz schlug kräftig.
"`Ich war mittlerweile krank."'
So matt sah das Gesicht des Mädchens aus.
\\ "`Wegen dem Regen von neulich, nicht wahr?"'
Sie nickte leise.
"`Bist du wieder gesund?"'
\\ "`Noch nicht \ldots"'
\\ "`Dann musst du im Bett bleiben."'
\\ "`Aber da drinnen war es so stickig\ldots\
So, an diesem Tag hatte ich viel Spaß\ldots\
Übrigens, ich weiß nicht, woher diese Flecken kommen.
Ich kann sie nicht entfernen."'
Sie sah an den vorderen Saum der rosa Strickjacke hinunter.
Da hatte sie braune Schlammflecken oder Ähnliches.
Sie bekam leise ein Grübchen und sagte:
"`Also, was denkst du, was ist das?"'
Er starrte nur den Jackensaum an.
"`Ich erinnere mich.
Da bin ich auf dich gestiegen,
als wir über den Graben gegangen sind, ja?
Dabei hat dein Rücken diese Flecken gemacht."'
Er fühlte, dass sein Gesicht plötzlich heiß wurde.
An der Weggabelung sagte sie:
"`Also, heute Morgen hat meine Familie rote Datteln geerntet.
Für einen Ritus morgen\ldots"'
Sie bot eine Handvoll Datteln an.
Er zögerte.
"`Probier diese aus!
Ich habe gehört, mein Urgroßvater hat sie gepflanzt.
Sie sind sehr süß."'
\\ Er streckte die eingerollten Hände vor und sagte dabei:
"`Wie groß die Früchte sind!"'
\\ "`Und kurz nach dem Ritus müssen wir das Haus hergeben."'
Schon bevor die Familie des Mädchens hierhin umzog,
hatte er von Erwachsenen gehört,
dass sie in die Heimat zurückkommen mussten,
weil das Unternehmen des Enkels von Herrn Yun in Seoul gescheitert war.
Anscheinend mussten sie dieses Mal außerdem das Haus in der Heimat weggeben.
"`Ich weiß nicht warum, aber ich will nicht wegziehen.
Trotzdem kann ich nicht die Erwachsenen davon abhalten\ldots"'
Wie nie zuvor zeigten die schwarzen Augen des Mädchens Einsamkeit.
Auf dem Rückweg nach dem Abschied von dem Mädchen,
kamen dem Jungen immer wieder ihre Worte, dass sie wegziehen würde.
% Darüber dürfte es eigentlich keine Enttäuschung noch Betrübnis geben.
Das wäre kein Grund für Enttäuschung und Betrübnis gewesen.
Jedoch schmeckten die süßen Datteln ihm überhaupt nicht.
Diese Nacht ging er heimlich aufs Walnussfeld von Herrn Deoksoe.
Er kletterte auf einen Baum, auf den am Tag er ein Auge geworfen hatte.
Er schlug mit einem Stab auf den Zweig, den er angeschaut hatte.
Die fallenden Walnüsse klangen besonders laut.
Er fröstelte vor Angst.
Im nächsten Moment aber rief er: "`Größe Walnüsse, fallt reichlich, fallt reichlich!"'
Er schlug mit dem Stab immer wieder mit einer Kraft unbekannten Ursprungs.
Auf dem Rückweg trat er nur auf die Schatten des Mondes der zwölften Nacht.
Er schätzte das erste Mal den Schatten.
Er berührte die gefüllte Tasche.
Er hatte keine Angst davor, dass
man eine Hautkrankheit bekommen kann, wenn
man Walnüsse mit bloßen Händen schält.
Er meinte einfach,
er sollte das Mädchen bald die lokal besten Walnüsse von Herrn Deoksoe
genießen lassen.
Dann dachte er: "`Donnerwetter!"'.
Er hatte dem Mädchen nicht gesagt,
sie solle einmal vor dem Umzug an den Bach herauskommen,
nachdem sie wieder gesund wäre.
"`Dumm, dumm."'
Am nächsten Tag als er von der Schule zurückkam,
fand er seinen Vater im Festkleid, ein Hähnchen im Arm. % haben.
Er fragte den Vater, wohin er ginge.
Ohne darauf zu antworten wog er das Hähnchen im Arm:
"`Geht das?"'.
\\ Die Mutter gab ihm einen Sack.
"`Schon einige Tage sagt es, gjal-gjal
und sucht dabei einen Ort, um Eier zu legen.
Es ist wohl fett genug, obwohl nicht groß."'
Dieses Mal fragte er die Mutter, wohin der Vater ginge.
"`Er geht zu Herrn Yun drüben im Seodang-gol\footnote{
"`Seodang"' ist eine veraltete Form der privaten Grundschule,
wo Kinder lernen,
Zeichen und Text zu lesen.
"`Gol"' bedeutet das Tal.
Wahrscheinlich war Herr Yun der Lehrer der "`Seodang"'.},
um für den Ritus zu spenden."'
\\ "`Warum nicht etwas Großes bringen, wie den fleckigen Hahn da\ldots"'
\\ Darüber lachte der Vater: "`Haha, mein Sohn!
Trotzdem ist dies gehaltreich."'
Aus keinem ersichtlichen Grund wurde der Junge verlegen.
Er warf die Tasche von sich und ging an den Kuhstall,
wo er einer Kuh auf den Rücken schlug,
als hätte er eine Bremse umgebracht.

Das Wasser des Bachs reifte Tag für Tag.\footnote{Der ursprüngliche
  koreanische Satz gibt auch wörtlich keinen Sinn.}
Der Junge ging nach unten von der Weggabelung.
Das Seodang-gol Dorf schien vom Ende des Schilffelds betrachtet
unter dem blauen Himmel viel näher.
Erwachsene sagten,
die Familie des Mädchens ziehe morgen nach Yang-pyeong.
Dort werde sie einen kleinen Laden haben.
Er fasste unbewusst die Walnüsse in der Tasche an und
brach dabei endlos mit der anderen Hand Schilfblüten ab.
In dieser Nacht kam ihm immer die gleiche Frage:
"`Gehe ich morgen zum Mädchen oder nicht, um sie wegziehen zu sehen?
Wenn ja, kann ich sie sehen oder nicht?"'
An einem Punkt merkte er, dass er eingeschlafen war.
\\ "`Wie schade!"'
Irgendwann war der Vater vom Dorfzentrum zurückgekommen.
"`Familie Yun macht so schlechte Erfahrungen.
Sie verkaufen so viel Feld,
geben das Haus her, wo sie seit Generationen gewohnt haben.
Dazu werden sie sogar noch mit dem furchtbaren Tod konfrontiert\ldots"'
\\ Unter der Lampe hatte die Mutter im Arm etwas zum Nähen.
"`Das Mädchen war der einzige Urenkel, ja?"'
\\ "`Ja, sie haben schon früher die zwei Jungen verloren\ldots"'
\\ "`Was für ein Pech mit ihren Kinder!"'
\\ "`Genau.
Dieses Mal konnten sie sogar nicht genug Medizin anwenden.
Jetzt hat die Familie keinen Nachfolger\ldots\
Übrigens, dieses Mädchen ist solch ein erwachsenes Kind,
denn sie hat vor dem Tod darum gebeten,
dass sie in ihrer gewöhnlichen Kleidung begraben wird,
wenn sie stirbt\ldots"'

\end{document}

Local Variables:
coding: utf-8
TeX-engine: xetex
TeX-PDF-mode: t
eval: (ispell-change-dictionary "de_DE")
eval: (TeX-source-correlate-mode t)
End:
