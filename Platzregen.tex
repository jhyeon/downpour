\documentclass[prd,12pt,tightenlines,notitlepage,nofootinbib]{revtex4-1}
%\usepackage[utf8]{inputenc}
\usepackage[ngerman]{babel}
%\usepackage{mathpazo}
\usepackage{fontspec}
\setmainfont[Ligatures=TeX]{TeX Gyre Pagella}
\setsansfont[Ligatures=TeX]{Arimo}
\setmonofont[Ligatures=TeX]{Cousine}
\fontsize{12}{12}
\linespread{2.4}
%\linespread{1}

\begin{document}

\title{Der Platzregen (1953)}
\author{Der koreanische Originaltext von Sun-won Hwang (1915-2000)
\\ Übersetzt ins Deutsch von Jae-hyeon Park}
% \\ Korrigiert von Jupp Hartmann }
\maketitle

\noindent
Als der Junge am Bach das Mädchen sah, erkannte er sofort es als die
Urenkelin von Herrn Yun\footnote{Im Originaltext heißt sein Titel "`Choshi"'.
"`Choshi"' bedeutet wörtlich "`Erste (Vorbereitungs)Prüfung"' und
auch eine Person, die sie bestand.
Die Choseon Dynastie hatte
die erste (lokale) und die zweite (zentrale) Vorbereitungsprüfung,
danach drei Stufen Hauptprüfungen fürs Auswahl der Beamten.
Eine regelmäßige "`Choshi"' fand alle drei Jahre statt und
wählte im ganzen Staat 1400 Personen aus.}.
Es bewegt seine Hände im Wasser und hat
Spaß daran, wie es solch einen Bach nie in Seoul sah.  Schon einige
Tage spielte es nach der Schule mit dem Wasser.  Bis gestern amüsierte
es sich am Bach, aber heute ist es mitten auf den Trittsteinen.  Er
setzte sich aufs Ufer, um zu warten, bis es weggeht.  Glücklich kam
jemand und es machte den Weg frei.

Am nächsten Tag ging der Junge ein bisschen später an den Bach aus.
Mitten auf den Trittsteinen sitzend hat das Mädchen dieses mal sein
Gesicht gewaschen.  Einfach weiß war sein Nacken aus der rosa
Strickjacke mit den hochgekrempelten Ärmeln.  Nachdem eine Zeit lang
es sein Gesicht wusch, sieht es ins Wasser hinein.  Es starrt wohl das
Abbild seines Gesichts an.  Plötzlich schöpft es mit den Händen
Wasser, wie ein kleiner Fisch vorbei schwimmt.  Es schöpft schnell nur
Wasser, ohne darauf zu achten, ob er auf dem Ufer sitzt.  Jedoch fängt
es immer wieder nichts.  Anscheinend macht es sich einen Spaß daraus,
nur das Wasser weiter zu schöpfen.  Vielleicht wird es weggehen, nur
wenn jemand den Bach überquert.  Dann nimmt es etwas aus dem Wasser.
Einen weißen Kieselstein.  Plötzlich steht es auf und hüpft über die
Trittsteine.  Auf der anderen Seite des Bachs dreht es sich um und
sagt: "`Dieser Dumme!"'  Der Kieselstein flog hierhin.  Der Junge stand
unbewusst abrupt auf.  Das Mädchen läuft mit dem flatternden kurzen
Haar.  Es trat auf den Weg durch ein Schilffeld.  Dahinter liegen nur
Schilfblüten glänzend in der Herbstsonne\footnote{In Korea ist
der Himmel im Herbst meistens klar.}.
Es werde bald aus dem
Schilffeld erscheinen, dachte er.  Eine lange Zeit sei vergangen.
Trotzdem erscheint es nicht.  Er stellte sich auf die Zehen.  Noch
ziemliche Zeit sei vergangen, meinte er.  Dort am Ende des
Schilffelds bewegte sich ein Strauß.  Das Mädchen hielt Schilfblüten
im Arm.  Es ging nun langsam.  Sein Haar mit einer Schilfblüte dazu
leuchtete vor der extrem klaren Herbstsonne.  Es schien, als wäre
nicht das Mädchen gegangen, sondern die Schilfblüte.  Der Junge blieb
so stehen, bis die Schilfblüte gar nicht mehr gesehen wurde.  Er sah
auf den Kieselstein hinunter, den es geworfen hatte.  Die Nässe darauf
verflüchtigte sich.  Er nahm den Kieselstein und stak ihn in die
Tasche.

Vom nächsten Tag ging der Junge noch später an den Bach aus.  Es gab
keinen Schatten des Mädchens, zum Glück.  Jedoch war es komisch, dass
irgendeine Ecke seines Herzens sich immer leerer anfühlte, als Tag für
Tag er das Mädchen nicht fand.  Es wurde seine Gewohnheit, den
Kieselstein in der Tasche zu berühren.  Eines solchen Tages setzte er
sich mitten auf die Trittsteine, worauf sitzend das Mädchen früher mit
dem Wasser Spaß gehabt hatte.  Er legte ins Wasser seine Hände ein.
Er wusch sein Gesicht.  Er sah ins Wasser hinein.  Das dunkelbraunes
Gesicht spiegelte sich.  Es gefiel ihm nicht.  Er schöpfte mit den
Händen das Gesicht im Wasser, immer wieder.  Zu einem Punkt wurde er
überrascht und stand auf.  Das Mädchen kommt über den Bach hierhin!
"`Es hat sich versteckt und gesehen, was ich getan habe."'
Er fing an,
zu laufen.  Er tat einen falschen Schritt auf den Trittstein.  Ein
Fuß fiel ins Wasser.  Er lief weiter.  Er wünschte sich irgendwo zu
verstecken.  Diese Seite hat kein Schilffeld, sondern Buchweizenfeld.
Er dachte, die Buchweizenblüten röchen in der Nase scharfer als je
zuvor.  Die Mitte der Stirn schmerzte.  Salzhaltige Flüssigkeit floss
in den Mund.  Nasenbluten.  Er rannte einfach und wischte dabei sich
mit einer Hand das Blut ab.  Es kam ihm vor, als hätte irgendwo die
Stimme "`dumm, dumm"' hinter ihm weiter hergelaufen.

Es war Samstag.  Am Bach fand der Junge, das Mädchen, das einige
Tage nicht erschien, hatte auf der anderen Seite sitzend mit dem Wasser
Spaß.  Er fing an, über die Trittsteine so zu gehen, als hätte er es
nicht gesehen.  Heute geht er vorsichtig über die Trittsteine, worauf
er bisher wie auf einem Boulevard lief, denn er machte neulich vor dem
Mädchen einen Fehler.  "`Hallo."'  Er tat so, als hätte er das nicht
gehört.  Er trat aufs Ufer.  "`Hallo, wie heißt diese Muschel?"'
Unwillkürlich drehte er sich um.  Sein Blick traf die klaren schwarzen
Augen des Mädchens.  Er senkte sofort den Blick auf seine Handfläche.
"`Seidenmuschel\footnote{Direkte Übersetzung vom Koreanischen Name.
  Der wissenschaftliche Name ist Peronidia Venulosa.}."'\\
"`Was für ein schöner Name!"'  Sie erreichten eine Weggabelung.
Von hier muss es etwa 6 Kilometer nach unten gehen, er muss
nach oben etwa 4 Kilometer.  Es hielt und sagte: "`Warst du über den
Berg da?"'  Es zeigte aufs Ende des Felds.\\
"`Nein."'\\
"`Gehen wir mal dorthin?
Ich langweile mich allein zu Tode, nachdem ich aufs Land
gezogen bin."'\\
"`Es ist weiter, als es scheint."'\\
"`Aber nicht zu weit.
Ich habe in Seoul ganz weit Ausflüge gemacht."'  Seine Augen kamen ihm
vor, als würden sie sofort sagen: "`dumm, dumm!"'  Sie traten auf den Weg
durch ein Reisfeld.  Sie gingen am Ort vom Ernten des Reis vorbei.
Da stand ein Strohmann.  Der Junge schüttelte das Strohseil.  Einige
Spatzen fliegen weg.  "`Ach, heute müsste ich früh nach Hause gehen, um
Spatzen im Nutzgarten zu beobachten"', erinnert er sich.\\
"`Wie lustig!"'
Es greift und schüttelt unbändig das Seil.  Der Strohmann schwankt und
tanzt weiter.  Es bekam ein Grübchen in seiner linken Wange.  Drüben
steht noch ein Strohmann.  Es läuft dorthin.  Er lief dahinter her,
als hätte er versucht, zu vergessen, dass an einem Tag wie heute, er
früh nach Hause gehen müsste, um im Haushalt zu helfen.  Er rennt
einfach am Mädchen vorbei.  Grashüpfer stoßen und jucken sein
Gesicht.  Der blaue klarste Herbsthimmel kreist in seinem Blick.
Schwindelig, denn der Adler, der Adler, der Adler kreist.  Er blickt
zurück.  Es schüttelt den Strohmann, an dem er soeben vorbeikam.  Er
schwankt mehr als der Vorige.  Es gab einen Graben am Ende des
Reisfelds.  Zunächst hüpfte das Mädchen darüber.  Davon bis am Fuße
des Berges war ein Feld.  Sie gingen an der Feldgrenze vorbei, wo
Sorghumbunde standen.  "`Was ist das da?"'\\
"`Beobachtungshütte."'\\
"`Schmeckt eine Melone hier gut?"'\\
"`Klar!  Melonen schmecken gut aber Wassermelonen schmecken besser."'\\
"`Ich wollte eine probieren."'  Er
trat ins Rettichfeld ein, wo Melonen erzeugt worden waren.\footnote{
Normalerweise erzeugt man auf dem gleichen Feld
im Sommer Melonen und im Winter Rettiche.}
Er hat
zwei Rettiche ausgezogen und gebracht.  Sie waren noch nicht reif.
Nachdem er die Blätter verdrehte und wegwarf, gibt er ihm einen
Rettich.  Er beißt ein Stück vom Kopf des Rettichs ab, schält den
Rettich mit dem Nagel, beißt ihn, wie er zeigt, dass man ihn
so isst.  Das Mädchen tat es gleich.  Jedoch isst es noch nicht drei
Bisse und sagt: "`Ach, schmeckt scharf und stinkt."'  Es wirft den
Rettich weg.
\\ "`Wirklich schmeckt das zu schlecht."'  Er warf ihn weiter
weg.  Der Berg wurde nahe.  Augen brannten vor dem Herbstlaub.
"`Toll!"'  Es lief zum Berg.  Dieses mal lief er nicht dahinter her.
Trotzdem pflückte er in kurzer Zeit mehr Blumen als das Mädchen.
"`Das ist eine Kamille, das ist Buschklee, das ist eine Ballonblume, \ldots"'
\\ "`Ich habe nicht gewusst, eine Ballonblume ist so schön.\footnote{
Im Essen findet man oft die Wurzel der Ballonblume.}
Mir gefällt
Violett!  \ldots\  Übrigens, wie heißt diese gelbe Blume, die wie ein
Sonnenschirm aussieht?"'\\
"`Patrinia."'  Es hält die Patrinia wie einen
Sonnenschirm, mit einem Grübchen in seinem etwas roten Gesicht.  Er
bracht noch einen Strauß Blüten.  Er sucht frische Zweige aus und
gibt sie ihm.  Doch sagt es: "`Wirf nichts weg!"'  Sie stiegen auf den
Gebirgskamm.  Auf der anderen Seite des Tals lagen einige Hütten
friedlich zusammen.  Sie setzten sich auf einen Fels, obwohl niemand
so sagte.  Es schien, ihre Nähe wurde besonders ruhig.  Nur die
brennende Herbstsonne verbreitete den Geruch des trocknenden Grases.
"`Welche Blüten sind dann sie da?"'  Pfeilwurzstämme waren auf einem
Steilhang mit den Blüten verwickelt.  "`Sie sehen wie Glyzinien aus.
Es gab große Glyzinien in meiner Schule in Seoul.  Die Blüten da
erinnern mich an meine Freunden, die mit mir unter Glyzinien Spaß
hatten."'  Es steht sich still auf und geht an den Hang.  Es versucht,
einen Stamm mit vielen Blüten abzuschneiden.  Er wird nicht leicht
geschnitten.  Es nimmt alle seine Kräfte zusammen aber rutscht aus.
Es fasste den Stamm.  Er wurde überrascht und lief dahin.  Es streckte
eine Hand aus.  Er hält seine Hand und hebt es auf.  Es tut ihm Leid,
dass er für es den Stamm nicht abschnitt.  Ein Bluttropfen lag auf
seinem rechten Knie.  Er legte seine Lippen auf die Wunde und fing an,
zu saugen.  Anscheinend bekam er irgendeine Idee.  Er steht abrupt auf
und läuft nach drüben.  Nach einer Weile kam er keuchend zurück.  "`Das
wird dich heilen."'  Er streicht Kiefernharz auf die Wunde.  Sofort
läuft er unten zu den Pfeilwurzstämmen und bringt einige mit vielen
Blüten, die er mit den Zähnen abschnitt.  Danach: "`Da ist ein Kalb.
Gehen wir mal dorthin!"'  Es war ein Gelbes Kalb.  Es hatte noch keinen
Nasenring.  Er hielt es im Zaum und tat, als hätte er es am Rücken
gekratzt.  Plötzlich sprang er darauf.  Es hüpft herum.
Das weiße Gesicht, die rosa Strickjacke, der Blaue Rock des
Mädchens mischen sich mit den Blüten in seinem Arm.  Alle sehen wie
ein großer Strauß Blüten aus.  "`Schwindelig, aber ich steige nicht
aus."'  Er war stolz, dass er konnte, was es nicht nachahmen kann.
"`Was macht ihr hier?"'  Ein Bauer kam durch Pampasgras herauf.  Der
Junge sprang aus dem Kalb.  Er hat Angst, dass der Bauer ihn schelten
wird, weil es die Taille schaden kann, auf das Kalb zu steigen.  Der
bärtige Bauer blickte jedoch über das Mädchen und band nur den Zaum
auf.  "`Eilt nach Hause!  Vielleicht kommt ein Platzregen."'  Wirklich
ist eine Platte dunkle Wolke schon da über ihren Köpfen.  Alle
Richtungen klingen plötzlich lärmender.  Ein Wind geht durch und
schüttelt dabei Blätter.  In einem Augenblick wurde die Nähe violett.
Während sie den Berg heruntergehen, erklingen Regentropfen
auf den Eichenblättern.
Die Tropfen waren groß.  Es lief kalt über den Nacken. % hinunter.
Sofort sperrten Regenstreifen die Sicht.  Sie fanden im
Regennebel eine Beobachtungshütte.  Darin suchten sie Schutz vor dem Regen.
Jedoch hatte die Hütte geneigte Säulen und ein zerrissenes
Dach.  Trotzdem fand er einen Raum, wo weniger Regen fiel, ließ es
darin eintreten.  Die Lippen des Mädchens wurden ganz Blau.
Es schüttelte immer wieder die Schultern.  Er zog seine Baumwolljacke
aus und wickelte sie ihm um die Schultern.  Als er so tat, hob es nur
leise den vom Regen feuchten Blick auf ihn.  Danach sondert es aus dem
im Arm gebrachten Strauß gebrochene Zweige mit verzerrten Blüten aus.
Der Regen fing an, zu fallen, sogar wo es steht.  Da war kein Schutz
mehr.  Er sieht hinaus und läuft anscheinend mit irgendeiner Idee zum
Sorghumfeld.  Er schaut in die gestellten Sorghumbunde und stellt mehr
Bunde darauf, die er von der Nähe bringt.  Er sieht wieder darin.
Danach winkt er hierhin.  Kein Regen fiel in die Bunde, aber es war
leider dunkel und eng.  Der Junge musste draußen vorn im Regen
sitzen.  Dabei stieg Dampf von seinen Schultern.  Das Mädchen
flüsterte ihm, er solle darin sitzen.  Er sagte, es gehe so.  Wieder
sagte es ihm, er solle drinnen.  Er konnte es nicht mehr vermeiden,
einzutreten und ging nach hinten.
Damit wurde gebrochen der Strauß im Arm des
Mädchens.  Trotzdem dachte es, das sei kein Problem.  Die Nase deckte
der Geruch aus dem vom Regen nasse Körper des Junges.  Jedoch wandte
es den Kopf nicht.  Es fühlte doch, dass die Körperwärme des Junges
ziemlich den zitternden Leib heizt.  Die Sorghumblätter hörten
plötzlich auf, Lärm zu machen.  Draußen wurde hell.  Sie kamen aus
den Sorghumbunden.  Gleich dahin vorn floss die Sonne blendend.  Sie
erreichten den Graben.  Da strömte viel mehr Wasser.  Das Wasser war
schlammig in ganz roten Farbe.  Sie konnten nicht darüber hüpfen.  Er
drehte dem Mädchen den Rücken.  Darauf stieg es ohne Bedenken.  Das
Wasser reichte ihm bis zu den hochgekrempelten Kniehosen.  Es hielt
sich an seinem Nacken fest und stieß dabei einen Schrei aus.  Schon
bevor sie an die andere Seite gelangen, wurde der Herbsthimmel ohne
einzige Wolke blau, als wäre er immer so klar gewesen.

Danach zeigte sich nicht mehr das Mädchen.
Er fand es nicht, obwohl er jeden Tag an den Bach herlief.
Auf der Schule beobachtete er in der Pause den Spielplatz.
Er blickte verstohlen in die sechste Mädchenklasse.
Trotzdem sah er es nicht.
Auch an diesem Tag kam er an den Bach und berührte
den weißen Kieselstein dabei.
Dann fand er das Mädchen auf diesem Ufer gesetzt.
Sein Herz schlug kräftig.
"`Ich war mittlerweile krank."'
So matt sah das Gesicht des Mädchens aus.
\\ "`Wegen dem Regen von neulich, nicht wahr?"'
Es nickte leise.
"`Bist du wieder gesund?"'
\\ "`Noch nicht \ldots"'
\\ "`Dann musst du im Bett bleiben."'
\\ "`Aber da drinnen war es so stickig\ldots\
So, an diesem Tag hatte ich viel Spaß\ldots\
Übrigens, ich weiß nicht, woher diese Flecke kommen.
Ich kann nicht sie entfernen."'
Es sieht auf den vorderen Saum der rosa Strickjacke hinunter.
Da hatte sie braune Schlammflecke oder Ähnliches.
Es bekam leise ein Grübchen und sagte:
"`Also, was denkst du, diese sind?"'
Er starrte nur den Jackensaum an.
"`Ich erinnere mich.
Da bin ich auf dich gestiegen,
als wir über den Graben gegangen sind, ja?
Dabei hat dein Rücken diese Flecke gemacht."'
Er fühlte, dass das Gesicht plötzlich heiß wurde.
An der Weggabelung sagte es:
"`Also, heute Morgen hat meine Familie Rote Datteln geerntet.
Für ein Ritus morgen\ldots"'
Es bietet eine Handvoll Datteln an.
Er zögert.
"`Probier diese aus!
Ich habe gehört, mein Urgroßvater hat sie gepflanzt.
Sie sind sehr süß."'
\\ Er streckt die eingerollten Hände vor und sagt dabei:
"`Wie groß die Früchte sind!"'
\\ "`Und kurz nach dem Ritus müssen wir das Haus hergeben."'
Schon bevor die Familie des Mädchens hierhin umzog,
hatte er von Erwachsenen gehört,
dass sie zur Heimat zurückkommen mussten,
weil das Unternehmen des Enkels von Herrn Yun in Seoul gescheitert war.
Anscheinend dieses mal müssen sie außerdem das Haus in der Heimat weggeben.
"`Ich weiß nicht warum, aber ich will nicht wegziehen.
Trotzdem kann ich nicht die Erwachsenen davon abhalten\ldots"'
Wie nie zuvor zeigten seine schwarze Augen Einsamkeit.
Auf dem Rückweg nach dem Abschied vom Mädchen,
kamen dem Junge immer wieder seine Worte, dass es wegziehen wird.
% Darüber dürfte es eigentlich keine Enttäuschung noch Betrübnis geben.
Das wäre kein Grund für Enttäuschung und Betrübnis gewesen.
Jedoch schmeckten die süßen Datteln ihm überhaupt nicht.
Diese Nacht ging er heimlich aufs Walnussfeld von Senior Deoksoe.
Er kletterte auf einen Baum, worauf am Tag er ein Auge geworfen hatte.
Er schlug mit einem Stab auf den Zweig, den er angeschaut hatte.
Die fallenden Walnüsse klangen besonders laut.
Er fröstelte vor Angst.
Im nächsten Moment aber: "`Größe Walnüsse, fallt viel, fallt viel!"'
Er schlug mit dem Stab immer wieder mit Kraft unbekannten Ursprungs.
Auf dem Rückweg trat er nur auf die Schatten vom Mond der zwölften Nacht.
Er schätzte das erste mal den Schatten.
Er berührte die gefüllte Tasche.
Er hatte keine Angst davor, dass
man Hautkrankheit bekommen kann, wenn
man Walnüsse mit bloßen Händen schält.
Er meinte einfach,
er sollte das Mädchen bald die lokal besten Walnüsse von Senior Deoksoe
genießen lassen.
Dann dachte er: "`Donnerwetter!"'.
Er hatte dem Mädchen nicht gesagt,
es solle einmal vor dem Umzug an den Bach herauskommen,
nachdem es wieder gesund werde.
"`Dumm, dumm."'
Am nächsten Tag als er von der Schule zurückkam,
fand er seinen Vater im Festkleid ein Hähnchen im Arm haben.
Er fragte dem Vater, wohin er ging.
Ohne darauf zu antworten wog er das Hähnchen im Arm:
"`Geht das?"'.
\\ Die Mutter gab ihm einen Sack.
"`Schon einige Tage sagt es, gjal-gjal
und sucht dabei einen Ort, um Eier zu legen.
Es ist wohl fett genug, obwohl nicht groß."'
Dieses mal fragte er der Mutter, wohin der Vater ging.
"`Er geht zu Herrn Yun drüben im Seodang-gol\footnote{
"`Seodang"' bedeutet eine veraltete Form der privaten Grundschule,
wo Kinder lernen,
Zeichen und Text zu lesen.
"`Gol"' bedeutet das Tal.
Wahrscheinlich war Herr Yun der Lehrer der "`Seodang"'.},
um fürs Ritus zu spenden."'
\\ "`Warum nicht ein Großes bringen, wie den fleckigen Hahn da\ldots"'
\\ Davon lachte der Vater: "`Haha, mein Sohn!
Trotzdem lohnt dies sich."'
Aus keinem ersichtlichen Grund wurde der Junge verlegen.
Er warf die Tasche von sich und ging an den Kuhstall,
wo er eine Kuh auf den Rücken schlug,
als hätte er eine Bremse umgebracht.

Das Wasser des Bachs reifte Tag für Tag.\footnote{Der ursprüngliche
  koreanische Satz gibt auch wörtlich keinen Sinn.}
Der Junge ging nach unten von der Weggabelung.
Das Seodang-gol Dorf, vom Ende des Schilffelds betrachtet,
schien unter dem blauen Himmel viel näher.
Erwachsene sagten,
die Familie des Mädchens ziehe morgen nach Yang-pyeong.
Dort werde sie einen kleinen Laden haben.
Er fasste unbewusst die Walnüsse in der Tasche an und
brach dabei endlos mit der anderen Hand Schilfblüten ab.
In dieser Nacht kam ihm immer die gleiche Frage:
"`Gehe ich morgen zum Mädchen oder nicht, um sie wegziehen zu sehen?
Wenn ja, kann ich es sehen oder nicht?"'
Zu einem Punkt fand er, dass er eingeschlafen war.
\\ "`Wie schade!"'
Irgendwann war der Vater vom Dorfzentrum zurückgekommen.
"`Familie Yun macht so schlechte Erfahrungen.
Sie verkaufen so viel Feld,
geben das Haus her, wo sie seit Generationen gewohnt haben.
Dazu werden sie sogar mit dem schlechten Tod konfrontiert\ldots"'
\\ Unter der Lampe hatte die Mutter im Arm etwas zum Nähen.
"`Das Mädchen war der einzige Urenkel, ja?"'
\\ "`Ja, sie haben schon früher die zwei Jungen verloren\ldots"'
\\ "`Was für kein Glück ihren Kinder!"'
\\ "`Genau.
Dieses mal konnten sie sogar nicht genug Medizin anwenden.
Für jetzt hat die Familie keinen Nachfolger\ldots\
Übrigens, dieses Mädchen ist solch ein erwachsenes Kind,
denn es hat vor dem Tod darum gebeten,
dass es sicher in seiner gewöhnlich angezogenen Kleidung gegraben wird,
wenn es sterbt\ldots"'

\end{document}

Local Variables:
coding: utf-8
TeX-engine: xetex
TeX-PDF-mode: t
eval: (ispell-change-dictionary "de_DE")
eval: (TeX-source-correlate-mode t)
End:
